%
% Developpement des coefficients de Bernstein
%

\documentclass{amsart}

%%%<
\usepackage{mathrsfs}
\usepackage[utf8]{inputenc}
\usepackage[T1]{fontenc}
\usepackage[french]{babel}
\usepackage{geometry}
\usepackage{verbatim}
%%%>

\begin{comment}
:Title: Developpement des coefficients de Bernstein
Destiné pour produire un svg
A compiler avec : latex EquDevelopB.tex && dvisvgm -n EquDevelopB.dvi
\end{comment}

\geometry{
  textheight=5.2cm
}

\begin{document}

  \pagestyle{empty}
  \pagenumbering{gobble}

  \centerline{$\qquad$ En utilisant le développement de ${\left({1-t}\right)}^{k}$:}

  \begin{equation*}
    {\left({1-t}\right)}^{k} = \sum_{j=0}^{k} {{\left({-1}\right)}^{j} \binom{k}{j} {t}^{j}}
  \end{equation*}

  \centerline{$\qquad$ On peut réécrire le coefficient de Bernstein ($B$).}

  \begin{equation*}
    \begin{split}
      {B}_{i}^{n} {\left({t}\right)} & = \sum_{j=0}^{n-i} {{\left({-1}\right)}^{j}} {\binom{n}{i}} {\binom{n-i}{j}} {t^{i+j}} \quad \text{puis} \\
      {B}_{i}^{n} {\left({t}\right)} & = \sum_{j=0}^{n-i} {{\left({-1}\right)}^{j}} {\frac{n!}{{i!}{j!}{\left({n-\left({i+j}\right)}\right)!}}} {{t}^{i+j}} \\
    \end{split}
  \end{equation*}

\end{document}