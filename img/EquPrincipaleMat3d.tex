%
% Equation principale de la courbe paramétrique de Bezier
% Version Matricielle pour un exemple en 3D
%

\documentclass{amsart}

%%%<
\usepackage{mathrsfs}
\usepackage[utf8]{inputenc}
\usepackage[T1]{fontenc}
\usepackage[french]{babel}
\usepackage{geometry}
\usepackage{verbatim}
%%%>

\begin{comment}
:Title: Equation principale de la courbe paramétrique de Bezier, Représentation matricelle en dimension 3
Destiné pour produire un svg
A compiler avec : latex EquPrincipaleMat3d.tex && dvisvgm -n EquPrincipaleMat3d.dvi
\end{comment}

\geometry{
  textheight=3.5cm
}

\begin{document}

  \pagestyle{empty}
  \pagenumbering{gobble}

  \begin{equation*}
    \begin{bmatrix}
      \mathscr{B}_{x}{\left(t_{0}\right)} & \mathscr{B}_{y}{\left(t_{0}\right)} & \mathscr{B}_{z}{\left(t_{0}\right)}
    \end{bmatrix}
    =
    \begin{bmatrix}
      B_{0}^{n}\left(t_{0}\right) & \cdots & B_{i}^{n}\left(t_{0}\right) & \cdots & B_{n}^{n}\left(t_{0}\right)
    \end{bmatrix}
    .
    \begin{bmatrix}
      {P_{x}}_{0} & {P_{y}}_{0} & {P_{z}}_{0} \\
      \vdots & \vdots & \vdots \\
      {P_{x}}_{i} & {P_{y}}_{i} & {P_{z}}_{i} \\
      \vdots & \vdots & \vdots \\
      {P_{x}}_{n} & {P_{y}}_{n} & {P_{z}}_{n}
    \end{bmatrix}
  \end{equation*}

\end{document}