%
% Création de la matrice partie 1
% Explicitation du glissement de la sommation
%

\documentclass{amsart}

%%%<
\usepackage[utf8]{inputenc}
\usepackage[T1]{fontenc}
\usepackage[french]{babel}
\usepackage{geometry}
\usepackage{verbatim}
%%%>

\begin{comment}
:Title: Création de la matrice des coefficients partie 1, glissement de la sommation
Destiné pour produire un svg
A compiler avec : latex EquCreateMat1.tex && dvisvgm -n EquCreateMat1.dvi
\end{comment}

\geometry{
  textheight=6cm
}

\begin{document}

  \pagestyle{empty}
  \pagenumbering{gobble}

  \begin{equation*}
    \sum_{j=0}^{n-i} {t}^{i+j} \quad \text{et} \quad \sum_{k=i}^{n} {t}^{k} \quad \text{sont équivalents}
  \end{equation*}

\end{document}