%
% Equation principale de la courbe paramétrique de Bezier
%

\documentclass{amsart}

%%%<
\usepackage{mathrsfs}
\usepackage[utf8]{inputenc}
\usepackage[T1]{fontenc}
\usepackage[french]{babel}
\usepackage{interval}
\usepackage{geometry}
\usepackage{verbatim}
%%%>

\begin{comment}
:Title: Equation principale de la courbe paramétrique de Bezier
Destiné pour produire un svg
A compiler avec : latex EquPrincipale.tex && dvisvgm -n EquPrincipale.dvi
\end{comment}

\geometry{
  textheight=4.9cm
}

\begin{document}

  \pagestyle{empty}
  \pagenumbering{gobble}

  \centerline{$\qquad$ Avec ${n+1}, {n} \geq 1$ points donnés ${P}_{i} \quad {i}\in\interval[scaled]{0}{n}$}

  \centerline{$\qquad$ Les points ${P}_{i}$ sont appelés points de contrôle, les ${B}_{i}^{n}$ sont appelés polynômes de Berstein.}

  \begin{equation*}
    \begin{split}
      \mathscr{B}\left({t}\right) & = \sum_{i=0}^{n} {{B}_{i}^{n}\left(t\right) {P}_{i}} \quad {t}\in\interval[scaled]{0}{1} \quad \text{avec} \\
      {B}_{i}^{n}\left({t}\right) & = \binom{n}{i}t^{i}{\left({1-t}\right)}^{n-i} \quad \text{et} \quad {\binom{n}{i} = \frac{n!}{\left(n-i\right)!i!}}
    \end{split} 
  \end{equation*}

\end{document}