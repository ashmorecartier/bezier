%
% Figure Etape 3 construction du Quart de Nonante
%

\documentclass{amsart}

%%%<
\usepackage{tikz}
\usepackage[utf8]{inputenc}
\usepackage[T1]{fontenc}
\usepackage[french]{babel}
\usepackage{geometry}
\usepackage{verbatim}
%%%>

\begin{comment}
:Title: Construction du quart de nonante, étape 3
Destiné pour produire un svg
A compiler avec : latex FigEtape3.tex && dvisvgm -n FigEtape3.dvi
\end{comment}

\geometry{
  textheight=4cm
}

\begin{document}

  \pagestyle{empty}
  \pagenumbering{gobble}

  \begin{tikzpicture}[scale=.5]
    \draw[step=1cm,blue,very thin] (-0.9,-0.9) grid (28.5,6.1); % grid
    \draw[thick,->] (0,0) -- (25,0) node[right] {Largeur}; % axe X
    \draw[thick,->] (0,0) -- (0,5.5) node[right] {Bouge}; % axe Y

    \draw[red,fill=red] (0,4) circle (4pt) node[left,color=black] {$b$}; % sommet du bouge
    \draw[red,fill=red] (1.5307cm+6cm-1cm,3.6955cm) circle (4pt); % point 1
    \draw[red,fill=red] (2.8284cm+12cm-2cm,2.8284cm) circle (4pt); % point 2
    \draw[red,fill=red] (3.6955cm+18cm-3cm,1.5307cm) circle (4pt); % point 3
    \draw[red,fill=red] (24,0) circle (4pt) node[below,color=black] {$l$}; % point demi-largeur

    \draw[blue] (0,4) .. controls (1.5307cm+6cm-1cm,4cm) and (2.8284cm+12cm-2cm,3.4cm) .. (24,0);

  \end{tikzpicture}

\end{document}