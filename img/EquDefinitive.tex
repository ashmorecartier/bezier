%
% Equation définitive de la courbe paramétrique de Bezier sous forme matricielle
%

\documentclass{amsart}

%%%<
\usepackage{mathrsfs}
\usepackage[utf8]{inputenc}
\usepackage[T1]{fontenc}
\usepackage[french]{babel}
\usepackage{geometry}
\usepackage{verbatim}
%%%>

\begin{comment}
:Title: Equation définitive de la courbe paramétrique de Bezier sous forme matricielle
Destiné pour produire un svg
A compiler avec : latex EquDefinitive.tex && dvisvgm -n EquDefinitive.dvi
\end{comment}

\geometry{
  textheight=4.9cm
}

\begin{document}

  \pagestyle{empty}
  \pagenumbering{gobble}

  \begin{equation*}
    \begin{bmatrix}
      \mathscr{B}{\left(t\right)}
    \end{bmatrix}
    =
    \begin{bmatrix}
      {t}^{0} & {t}^{1} & \cdots & {t}^{n}
    \end{bmatrix}
    .
    \begin{bmatrix}
      {C}_{0,0}^{n}   &             0   & \cdots & 0                 &             0 \\
      {C}_{0,1}^{n}   & {C}_{1,1}^{n}   & \cdots & 0                 &             0 \\
      \vdots          & \vdots          & \vdots & \vdots            &        \vdots \\
      {C}_{0,n-1}^{n} & {C}_{1,n-1}^{n} & \cdots & {C}_{n-1,n-1}^{n} &             0 \\
      {C}_{0,n}^{n}   & {C}_{1,n}^{n}   & \cdots & {C}_{n-1,n}^{n}   & {C}_{n,n}^{n}
    \end{bmatrix}
    .
    \begin{bmatrix}
      {P}_{0} \\
      \vdots \\
      {P}_{i} \\
      \vdots \\
      {P}_{n}
    \end{bmatrix}
  \end{equation*}
  \begin{equation*}
    \quad \text{avec} \quad C_{i, k}^{n} = {\left(-1\right)}^{k-i} {\frac{n!}{i!\left({k-i}\right)!\left({n-k}\right)!}} \quad \text{si} \quad i \leq k \leq n
  \end{equation*}

\end{document}